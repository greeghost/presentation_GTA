\documentclass[aspectratio=169]{beamer}
\usepackage{amsmath, amssymb}
\usepackage{graphicx}
\usepackage{xcolor}
\usepackage{tikz}
\usetikzlibrary{arrows.meta}
\usetikzlibrary{positioning}
\tikzset{
    sommet/.style = {circle, draw, on grid},
    >=Latex
}
\usepackage[boxed]{algorithm2e}
\SetKwProg{Algo}{algorithm}{}{}


% Metropolis theme
\usetheme{metropolis}
\usefonttheme[onlymath]{serif}
\metroset{progressbar=frametitle}
\metroset{titleformat=smallcaps}
\metroset{numbering=fraction}
\metroset{block=fill}
\definecolor{mypink}{HTML}{e91e63}
\setbeamercolor{alerted text}{fg=mypink}

\makeatletter
\setlength{\metropolis@titleseparator@linewidth}{1pt}
\setlength{\metropolis@progressonsectionpage@linewidth}{1.5pt}
\setlength{\metropolis@progressinheadfoot@linewidth}{2.5pt}
\makeatother

\title{Combinatorial Auctions with Restricted Complements}
\subtitle{An article by I. Abraham, M. Babaioff, S. Dughmi and T. Roughgarden}
\author{Hugo \textsc{Boulier}, Hugo \textsc{Francon} \& Igor \textsc{Martayan}}
\date{November 8, 2021}

\begin{document}

{\metroset{background=dark}\maketitle}

\begin{frame}{Introduction}
    \begin{columns}
        \column{0.6\textwidth}
        \begin{itemize}
            \item Auction with $n$ bidders and $m$ items
            \item Some item groups gain added value when purchased together
            \item Use of hyperedges to model these preferences
        \end{itemize}

        \column{0.4\textwidth}
        \begin{figure}[H]
            \begin{tikzpicture}[node distance=0.25cm and 0.25cm, font=\small]
                \node (complet) {10};
                \node (galette) [label={0.5}, above left=of complet] {\includegraphics[width=0.4\textwidth]{img/crepe.jpg}};
                \node (oeuf) [label={2.5}, above right=of complet] {\includegraphics[width=0.4\textwidth]{img/oeuf.jpg}};
                \node (jambon) [label={1.5}, below left=of complet] {\includegraphics[width=0.4\textwidth]{img/jambon.png}};
                \node (fromage) [label={2.5}, below right=of complet] {\includegraphics[width=0.4\textwidth]{img/fromage.jpg}};
                \draw (galette) -- (complet) -- (jambon) -- (complet) -- (oeuf) -- (complet) -- (fromage);
            \end{tikzpicture}
        \end{figure}
    \end{columns}
\end{frame}

\begin{frame}{Core Questions}
    \begin{itemize}
        \item Polynomial-time approximation maximizing welfare
        \item A generalization of this approach
        \item A truthful approximation mechanism
    \end{itemize}
\end{frame}

{\metroset{background=dark}\section{Polynomial-time approximation maximizing welfare}}

\begin{frame}{Polynomial-time approximation maximizing welfare}
    \begin{columns}
        \column{0.6\textwidth}
        G is a planar graph

        \begin{block}{theorem}
            For any $\varepsilon > 0$, we can compute a $(1 + \varepsilon)$-approximation of the optimal welfare repartition in polynomial-time.
        \end{block}

        \begin{block}{theorem}
            Once combined with VCG auctions, this mechanism is truthful.
        \end{block}

        \column{0.4\textwidth}
        On se restreint aux graphe auquel on a enlevé des neouds à une certaine distance (variable) de la racine

        Pour chacun de ces graphes, on calcule les allocations, et pour l'un d'entre eux on a une (1+varepsilon)-approximation


    \end{columns}
\end{frame}

\begin{frame}[standout]
    Can we generalize this approach ?
\end{frame}

\begin{frame}{Approximate welfare maximization algorithm}
    \begin{columns}
        \column{0.6\textwidth}
        We generalize to $r$-hypergraph with \alert{nonnegative hyperedge weight}

        \begin{block}{Theorem}
            The welfare maximization problem can be \(r\)-approximated in polynomial time.
        \end{block}

        \column{0.4\textwidth}
        2 step algorithm:
        \begin{itemize}
            \item (real) linear programming
            \item randomized assignment
        \end{itemize}
    \end{columns}
\end{frame}

\begin{frame}{The linear program}
    \[
        \text{maximize:} \sum_{i \,\in \{\text{bidders}\}} \left(
            \sum_{j \,\in \{\text{items}\}} w_{ij}\, {\color{orange} x_{ij}}
            + \sum_{e \,\in \{\text{edges}\}_i} w_{ie}\, {\color{blue} z_{ie}}
        \right)
    \]
    with
    \begin{align*}
        \sum_{i \,\in \{\text{bidders}\}} {\color{orange} x_{ij}}  = 1 & \hspace{.5cm} \text{for all item \(j\)}\\
        {\color{blue} z_{ie}}  \leqslant {\color{orange} x_{ij}} & \hspace{.5cm} \text{for all player \(i\), edge \(e\) and item \(j\)}\\
        {\color{orange} x_{ij}}   \geqslant 0 & \hspace{.5cm} \text{for all player \(i\) and item \(j\)}\\
        {\color{blue} z_{ie}}  \geqslant 0 & \hspace{.5cm} \text{for all player \(i\) and edge \(e\)}
    \end{align*}

    $\Rightarrow$ Real-valued valuation in \alert{polynomial time}.
\end{frame}

\begin{frame}{The randomized rounding algorithm}
    \begin{columns}
        \column{0.6\textwidth}
        \hrule
        \vspace{.1cm}
        \textbf{Input:} an optimal (real) solution \((x^*, z^*)\) of the linear program

        \vspace{.5cm}
        \textbf{While} there exists an unassigned item
        \begin{enumerate}
            \item choose a player \(i\) at random
            \item choose a threshold \(t \in [0, 1]\) at random
            \item assign to \(i\) every unassigned item with \(x_{ij}^* \geqslant t\)
        \end{enumerate}
        \hrule

        \column{0.4\textwidth}
    \end{columns}
\end{frame}

\begin{frame}[standout]
    Can we solve this problem ?
\end{frame}

\begin{frame}{A truthful approximation mechanism}
    \begin{columns}
        \column{0.6\textwidth}
        ...

        \column{0.4\textwidth}
        ...
    \end{columns}
\end{frame}

\begin{frame}{Conclusion}
    ...
\end{frame}

\end{document}
