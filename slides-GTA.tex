\documentclass[aspectratio=169]{beamer}
\usepackage{appendixnumberbeamer}
\usepackage{amsmath, amssymb}
\usepackage{xcolor}
\usepackage{tikz}
\usetikzlibrary{arrows.meta}
\usetikzlibrary{positioning}
\tikzset{>=Latex}

% Metropolis theme
\usetheme{metropolis}
\usefonttheme[onlymath]{serif}
\metroset{progressbar=frametitle}
\metroset{titleformat=smallcaps}
\metroset{numbering=fraction}
\metroset{block=fill}
\definecolor{mypink}{HTML}{e91e63}
\setbeamercolor{alerted text}{fg=mypink}

\makeatletter
\setlength{\metropolis@titleseparator@linewidth}{1pt}
\setlength{\metropolis@progressonsectionpage@linewidth}{1.5pt}
\setlength{\metropolis@progressinheadfoot@linewidth}{2.5pt}
\makeatother

\title{Combinatorial Auctions with Restricted Complements}
\subtitle{Un article de I. Abraham, M. Babaioff, S. Dughmi et T. Roughgarden}
\author{Hugo \textsc{Boulier}, Hugo \textsc{Francon} et Igor \textsc{Martayan}}
\date{8 novembre 2021}

\begin{document}

{\metroset{background=dark}\maketitle}

\begin{frame}{Introduction}
    \begin{itemize}
        \item Auction with $n$ bidders and $M$ items
        \item Some item groups gain added value when purchased together
        \item Use of hyperedges to model these preferences
    \end{itemize}
\end{frame}

\begin{frame}{Core Questions}
    \begin{itemize}
        \item Polynomial-time approximation maximizing welfare
        \item A generalization of this approach
        \item A truthful approximation mechanism
    \end{itemize}
\end{frame}

\section{Polynomial-time approximation maximizing welfare}

\begin{frame}{Polynomial-time approximation maximizing welfare}
    \begin{columns}
        \column{0.66\textwidth}
        G is a planar graph 

        \begin{block}{theorem}
            For any $\varepsilon > 0$, we can compute a $(1 + \varepsilon)$-approximation of the optimal welfare repartition in polynomial-time.
        \end{block}
            
        \begin{block}{theorem}
            Once combined with VCG auctions, this mechanism is truthful.
        \end{block}
        
        \column{0.34\textwidth}
        algo
    \end{columns}
\end{frame}

\begin{frame}[standout]
    Can we generalize this approach ?
\end{frame}

\begin{frame}{Approximate welfare maximization algorithm}
    \begin{columns}
        \column{0.66\textwidth}
        We generalize to $r$-hypergraph
        
        \begin{block}{theorem}
            ...
        \end{block}
        
        Unfortunately, this mechanism is not truthful.

        \column{0.34\textwidth}
        ...
    \end{columns}
\end{frame}

\begin{frame}[standout]
    Can we solve this problem ?
\end{frame}

\begin{frame}{A truthful approximation mechanism}
    \begin{columns}
        \column{0.66\textwidth}
        ...
        
        \column{0.34\textwidth}
        ...
    \end{columns}
\end{frame}

\begin{frame}{Conclusion}
    ...
\end{frame}

% \appendix
% Annexes ici

\end{document}
