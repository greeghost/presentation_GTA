\documentclass[aspectratio=169]{beamer}
\usepackage{appendixnumberbeamer}
\usepackage{amsmath, amssymb}
\usepackage{xcolor}
\usepackage{tikz}
\usepackage{algorithm2e}
\usetikzlibrary{arrows.meta}
\usetikzlibrary{positioning}
\tikzset{
    sommet/.style = {circle, draw, on grid},
    >=Latex
}

% Metropolis theme
\usetheme{metropolis}
\usefonttheme[onlymath]{serif}
\metroset{progressbar=frametitle}
\metroset{titleformat=smallcaps}
\metroset{numbering=fraction}
\metroset{block=fill}
\definecolor{mypink}{HTML}{e91e63}
\setbeamercolor{alerted text}{fg=mypink}

\makeatletter
\setlength{\metropolis@titleseparator@linewidth}{1pt}
\setlength{\metropolis@progressonsectionpage@linewidth}{1.5pt}
\setlength{\metropolis@progressinheadfoot@linewidth}{2.5pt}
\makeatother

\title{Combinatorial Auctions with Restricted Complements}
\subtitle{Un article de I. Abraham, M. Babaioff, S. Dughmi et T. Roughgarden}
\author{Hugo \textsc{Boulier}, Hugo \textsc{Francon} et Igor \textsc{Martayan}}
\date{8 novembre 2021}

\begin{document}

{\metroset{background=dark}\maketitle}

\begin{frame}{Introduction}
    \begin{columns}
        \column{0.66\textwidth}
        \begin{itemize}
            \item Auction with $n$ bidders and $M$ items
            \item Some item groups gain added value when purchased together
            \item Use of hyperedges to model these preferences
        \end{itemize}

        \column{0.34\textwidth}
            \begin{figure}[H]
                \begin{tikzpicture}[node distance=0.25cm and 0.25cm, font=\small]
                    \node (tout) [] {10};
                    \node (galette) [below right=of tout] {galette ; 1};
                    \node (jambon) [above right=of tout] {jambon ; 2};
                    \node (fromage) [below left=of tout] {fromage ; 2};
                    \node (oeuf) [above left=of tout] {oeuf ; 2};
                    \draw (galette) -- (tout) -- (jambon) -- (tout) -- (oeuf) -- (tout) -- (fromage);
                \end{tikzpicture}
            \end{figure}

    \end{columns}
    
    
\end{frame}

\begin{frame}{Core Questions}
    \begin{itemize}
        \item Polynomial-time approximation maximizing welfare
        \item A generalization of this approach
        \item A truthful approximation mechanism
    \end{itemize}
\end{frame}

\section{Polynomial-time approximation maximizing welfare}

\begin{frame}{Polynomial-time approximation maximizing welfare}
    \begin{columns}
        \column{0.66\textwidth}
        G is a planar graph 

        \begin{block}{theorem}
            For any $\varepsilon > 0$, we can compute a $(1 + \varepsilon)$-approximation of the optimal welfare repartition in polynomial-time.
        \end{block}
            
        \begin{block}{theorem}
            Once combined with VCG auctions, this mechanism is truthful.
        \end{block}
        
        \column{0.34\textwidth}
        On se restreint aux graphe auquel on a enlevé des neouds à une certaine distance (variable) de la racine
        
        Pour chacun de ces graphes, on calcule les allocations, et pour l'un d'entre eux on a une (1+varepsilon)-approximation
        
        
    \end{columns}
\end{frame}

\begin{frame}[standout]
    Can we generalize this approach ?
\end{frame}

\begin{frame}{Approximate welfare maximization algorithm}
    \begin{columns}
        \column{0.66\textwidth}
        We generalize to $r$-hypergraph
        
        \begin{block}{theorem}
            ...
        \end{block}
        
        Unfortunately, this mechanism is not truthful.

        \column{0.34\textwidth}
        ...
    \end{columns}
\end{frame}

\begin{frame}[standout]
    Can we solve this problem ?
\end{frame}

\begin{frame}{A truthful approximation mechanism}
    \begin{columns}
        \column{0.66\textwidth}
        ...
        
        \column{0.34\textwidth}
        ...
    \end{columns}
\end{frame}

\begin{frame}{Conclusion}
    ...
\end{frame}

% \appendix
% Annexes ici

\end{document}
